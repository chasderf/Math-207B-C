\documentclass{article}
\usepackage{amsmath}
\newcommand\tab[1][1cm]{\hspace*{#1}}
\begin{document}


\title {Math 207B Ordinary Differential Equations: Ch: 1 Introduction to PDE}

\author{Charlie Seager}

\maketitle

\textbf {Chapter 1.1 Paritial Differential Equations} A partial differential equation (PDE) is an equation involving an unknown function of two or more variables and certain of its partial deriviatives. 

\textbf{Definition} An expression of the form \\
(1) \tab $F(D^k u(x), D^{k-1} u(x),..., Du(x), u(x), x) = 0$ \tab $(x \in U)$ \\
is called a kth-order partial differential equation, where
\begin{center}
$F: R^{n^k} x R^{n^{k-1}} x \dots R^n x R x U \to R$
\end{center}
is given and 
\begin{center}
$u : U \to R$
\end{center}
is the unknown.

\textbf {Definitions} \\
(i) The partial differential equation (1) is called linear if it has the form
\begin{center}
$\sum{|\alpha| \leq k} a_{\alpha}(x) D^{\alpha}u = f(x)$
\end{center}
for given functions $a_{\alpha}(|\alpha| \leq k), f$. This linear PDE is homogeneous if f = 0. \\
(ii) The PDE (1) is semilinear if it has the form
\begin{center}
$\sum_{|\alpha|=k} a_\alpha (x) D^{\alpha} u + a_0 (D^{k-1}u,...,Du,u,x) = 0$
\end{center}
(iii) The PDE (1) is fully nonlinear if it depends nonlinearly upon the highest order derivatives. 

\textbf {Definition} An expression of the form \\
(2) \tab $F(D^k u(x), D^{k-1}u(x),...,Du(x), u(x), x) = 0 \tab (x \in U)$
is called a kth-order system of partial differential equations, where
\begin{center}
$F: R^{{mn}^k} x R^{{mn}^{k-1}} x \dots x R^{mn} x R^m x U \to R^m$
\end{center}
is given and 
\begin{center}
$u : U \to R^m, u = (u^1 ,..., u^m)$
\end{center}
is the unknown

\textbf {1.2 Examples} These are linear equations: Laplaces equation; Helmholtz (or eigenvalue) equation; Linear transport equation; Liouvill'es equation; heat (or diffusion) equation; Schrodinger's equation; Kolmogrorov's equation; Fokker-Planck equation; Wave equation; Klein-Gordon equation; Telegraph equation; General wave equation; Airy's equation; Beam Eqaution. \\
These are Nonlinear equations: Eikonal equation; nonlinear poisson equation; p-laplacian equations; minimal surface equation; Monge-Ampere equation; Hamilton-Jacobi equation; Scalar conservation law; Inviscid Burgers' equation; Scalar reaction-diffusion equation; Porous medium equation; nonlinear wave equation; kortewig-deVries (KdV) equation; nonlinear shrodinger equation; \\
These are linear systems: Equilibrium equations of linear elasticity; evolution equations of linear elasticity; maxwells equations; \\
These are nonlinear systems: System of conservation laws; reaction-diffusion system; Eulers equations for incompressible, invscid flow; Navier-Stokes equations for incompressible, viscous flow.

\textbf {Chapter 1.3 Strategies for Studying PDE}

\textbf {Chapter 1.4 Overview}




















\end{document}