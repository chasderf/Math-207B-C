\documentclass{article}
\usepackage{amsmath}
\usepackage{mathrsfs}  
\newcommand\tab[1][1cm]{\hspace*{#1}}
\begin{document}


\title {Math 207B Partial Differential Equations: Ch 4 Techniques for Solving important examples: Other ways to represent solutions}

\author{Charlie Seager}

\maketitle

\textbf {Turing instability}

\textbf {Chapter 4.2 Similarity Solutions} \\
When investigating partial differential equations, it is often profitable to
look for specific solutions u, the form of which reflects various symmetries
in the structure of the PDE. We have already seen this idea in our derivation
of the fundamental solutions for Laplace's equation and the heat equation
in §2.2.1 and §2.3.1 and our discovery of rarefaction waves for conservation
laws in §3.4.4. Following are some further applications of this important
method.

\textbf {Chapter 4.3 Tranformation Methods} \\
In this section we develop some of the theory for the Fourier transform $\mathcal{F}$ , the Radon transform $\mathcal{R}$ and the Laplace transform $\mathcal{L}$. These provide extremely powerful tools for converting certain linear partial differential equations intoeither algebraic equations or else differential equations involving fewer
variables.

\textbf {Chapter 4.3.1 Fourier Transform} In this section all functions are complex-valued, and - denotes the complex conjugate

\textbf {Definition} If $u \in L^1 (R^n)$, we define its Fourier transform $\mathcal{F} u = \hat{u}$ by 
\begin{center}
$\hat{u}(y) := \frac{1}{(2\pi)^{n/2}} \int_{R^n} e^{-ixy}u(x) dx \tab (y \in R^n)$
\end{center}
and its inverse Fourier tranform $\mathcal{F}^{-1} u = \hat{u}$ by 
\begin{center}
$\widehat{u}(y) := \frac{1}{(2\pi)^{n/2}} \int_{R^n} e^{ixy} u(x) dx \tab (y \in R^n)$
\end{center}

\textbf {THeorem 1 (Plancherel's THeorem)}. Assume $u \in L^1 (R^n) \cap L^2(R^n)$. Then $\hat{u}, \widehat{u} \in L^2(R^n)$ and
\begin{center}
$||\hat{u}||_{L^2(R^n)} = ||\widehat{u}||_{L^2(R^n)} = ||u||_{L^2(R^n)}$
\end{center}

\textbf {Theorem 2 (Properties of Fourier Transform)} Assume $u, v \in L^2(R^n)$. Then \\
\tab (i) $\int_{R^n} u, \hat{v}dx = \int_{R^n}, \hat{u}, \bar{\hat{v}} dy$ \\
\tab (ii) $(D^{\alpha}u)\hat{} = (iy)^{\alpha} \hat{u}$ \tab for each multiindex $\alpha$ such that $D^{\alpha}u \in L^2 (R^n)$. \\
\tab (iii) If $u, v \in L^1(R^n) \cap L^2(R^n)$, then $(u * v)\hat{} = (2\pi)^{n/2}\hat{u}\hat{v}$ \\
\tab (iv) Furthermore, $u = (\hat{u})\widehat{}$.
Assertion (iv) is the Fourier inversion formula which represents a function u in terms of the exponential plane waves $e^{ixy}$, provided $\hat{u} \in L^1(R^n)$
\begin{center}
$u(x) = \frac{1}{(2\pi)^{n/2}} \int_{R^n} e^{ixy} \hat{u} (y) dy$
\end{center}

\textbf {Definition} The Radon tranform $\mathcal{R}u = \tilde{u}$ of a function $u \in C_c^{\infty}(R^n)$ is 
\begin{center}
$\tilde{u}(x, \omega) := \int{\prod (s, \omega)} udS \tab (s \in R, \omega \in S^{n-1})$
\end{center}
The term on the right is the integral over the plane $\prod (s \omega)$ with respect to (n-1)-dimensional surface measure

\textbf {Theorem 3 (Properties of Radon tranform)} Assume $u \in C_c^{\infty}(R^n)$ Then  \\
\tab (i) $\tilde{u}(-s, -\omega) = \tilde{u}(s, \omega)$ \\
\tab (ii) $(D^{\alpha}u)^{\tilde{}} = \omega^{\alpha} \frac{\partial^{|\alpha|}}{\partial s^{|\alpha|}} \tilde{u}$ for each multiindex $\alpha$ \\
\tab (iii) $(\triangle u)^{\tilde{}} = \omega^{\alpha} \frac{\partial^2}{\partial s^2} \tilde{u}$ \\
\tab If u = 0 in $R^n - B(0, \mathcal{R})$, then $\tilde{u}(s, \omega) = 0$ for $|s| \geq \mathcal{R}$

\textbf {Theorem 4 (Radon and Fourier tranforms)} Assume that $u \in C_c^\infty (R^n)$. Then 
\begin{center}
$\bar{u}(r, \omega) := \int_R \tilde{u}(s, \omega)e^{-irs}ds = (2\pi)^{n/2} \hat{u}(r\omega) \tab (r \in R, \omega \in S^{n-1})$
\end{center}
where $\hat{u} = \mathcal{F}u$ is the Fourier tranform.

\textbf {Theorem 5 (Inverting the Radon tranform)} \\ \tab (i) We have 
\begin{center}
$u(x) = \frac{1}{2(2\pi)^n} \int_R \int_{S^{n-1}} \bar{u}(r, \omega)r^{n-1} e^{ir\omega \cdot x} dSdr$
\end{center}
the function $\bar{u}$ defined by (31) \\
\tab (ii) If n = 2k + 1 is odd, then 
\begin{center}
$u(x) = \int_{S^{n-1}} r(x \cdot \omega, \omega) dS$
\end{center}
for 
\begin{center}
$r(s, \omega) := \frac{(-1)^k }{2(2\pi)^{2k}}\frac{\partial^{2k}}{\partial s^{2k}} \tilde{u}(s, \omega)$.
\end{center}
Formulas (33) and (34) provide an elegant and useful decommpositions of u into plane waves.

\textbf {Definition} If $u \in L^1(R_+)$, we define its Laplace tranform $\mathcal{L}u = u \#$ to be 
\begin{center}
$u \#(s) := \int_0^\infty e^{-st} u(t) dt \tab (s \geq 0)$
\end{center}

\textbf {Chapter 4.4 Converting Nonlinear into Linear PDE}

\textbf {Chapter 4.5 Asymptotics}

\textbf {Chapter 4.6 Power Series}

\textbf {Definition} We say the surface $\Gamma$ is noncharacteristic for the partial differential equation (1) provided
\begin{center}
$\sum_{|\alpha|=k} a_{\alpha}\nu^{\alpha} \neq 0$ \tab on $\Gamma$
\end{center}
for all values of the arguments of the coefficients $a_{\alpha} (|\alpha| = k )$

\textbf {Theorem 1 (Cauchy data and Noncharacteristic surfaces)} Assume that $\Gamma$ is noncharacteristic for the PDE (1). Then if u is a smooth solution of (1) and u satisfies the Cauchy conditions (2), we can uniquely compute all the partial differential equations of u along $\Gamma$ in terms of $\Gamma$, the function $g_0, ...,g_{k-1}$ and the coefficients $a_{\alpha} (|\alpha| = k), a_0$

\textbf {Definition} A function $f: R^n \to R$ is called (real) analytic near $x_0$ if there exist $r > 0$ and constants $\{f_{\alpha}\}$ such that 
\begin{center}
$f(x) \sum_{\alpha} f_{\alpha} (x-x_0)^{\alpha} \tab (|x-x_0| < r)$,
\end{center}
the sum taken over all multiindex $\alpha$

\textbf {Definition} Let 
\begin{center}
$f = \sum_{\alpha} f_{\alpha} x^{\alpha}, g = \sum_{\alpha} g_{\alpha}x^{\alpha}$
\end{center}
be two power series. We say g majorizes f, written
\begin{center}
$g >> f$
\end{center}
provided 
\begin{center}
$g_{\alpha} \leq |f_{\alpha}$ \tab for all multiindices $\alpha$
\end{center}

\textbf {Theorem 2 (Cauchy-Kovalevskaya Theorem)} Assume $\{B_j\}_{j=1}^{n-1}$ and c are real analytics functions. Then there exist $r > 0$ and a real analytic function
\begin{center}
$u = \sum_{\alpha}u_{\alpha}x^{\alpha}$
\end{center}
solving the boundary-value problem (15)























\end{document}