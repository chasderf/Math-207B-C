\documentclass{article}
\usepackage{amsmath}
\newcommand\tab[1][1cm]{\hspace*{#1}}
\begin{document}

\title {Math 207C Partial Differential Equations: Ch 6 Second-Order Eliptic Equations}

\author{Charlie Seager}

\maketitle

\textbf {Chapter 6.1 Definitions}

\textbf {Chapter 6.1.1 Elliptic equations} \\ In this chapter we will mostly study the boundary-value problem
http://users.uoa.gr/~pjioannou/nonlin/Strogatz,
\begin{center}
$\{ Lu = f$ \tab in U \\
$ \{ u = 0$ \tab on $\partial U$
\end{center}
where U is an open, bounded subset of $R^n$ and $u: \bar{U} \to R$ is the unknown u = u(x). Here $f: U \to R$ is given and L denotes a second order partial differential operator having either the form
\begin{center}
(2) \tab $Lu = - \sum_{i,j=1}^n (a^{ij}(x)u_{x_{i}})_{xj} + \sum_{i=1}^n b^i (x) u_{x_{i}} + c(x) u$
\end{center}
or else 
\begin{center}
(3) \tab $Lu = - \sum_{i,j=1}^n a^{ij} u_{x_{i}x_{j}} + \sum_{i=1}^n b^i (x) u_{x_{i}} + c(x) u,$
\end{center}
for given coefficient functions $a^{ij}, b^i, c(i,j=1,..,n)$. \\
We say that the PDE Lu = f is in divergence form if L is given by (2) and is in nondivergence form provided by (3). The requirement that u = 0 on $\partial U$ in (1) is sometimes called Dirichlet's boundary condition.

\textbf {Definition} We say the partial differential operator L is (uniformly) elliptic if there exists a constant $\theta > 0$ such that
\begin{center}
(4) \tab $\sum_{i,j=1}^n a^{ij}(x) \xi_i \xi_j \geq \theta |\xi|^2$
\end{center}
for a.e. $x \in U$ and all $\xi \in R^n$

\textbf {Definitions} (i) The bilinear form B[u,v] associated with the divergence form eliptic operator L defined by (2) is
\begin{center}
(8) \tab $B[u,v] := \int_U \sum_{i,j=1}^n a^{ij} u_{x_{i}}, v_{x_{j}} + \sum_{i=1}^n b^i u_{x_{i}} v + cuv dx$
\end{center}
for $u, v \in H_0^1 (U)$ \\
\tab (ii) We say that $u \in H_0^1 (U)$ is a weak solution of the boundary value problem (1) if 
\begin{center}
$B[u,v] = (f,v)$
\end{center}
for all $v \in H_0^1(U)$ where ( , ) \{the book just leaves the interval blank btw \} denotes the inner product in $L^2(U)$

\textbf {Definition} We say $u \in H_0^1(U)$ is a weak solution of problem (10) provided 
\begin{center}
$B[u,v] = \langle f, v \rangle$
\end{center}
for all $v \in H_0^1(U)$ where $\langle f, v \rangle = \int_U f^0 v + \sum_{i=1}^n f^i v_{x_{i}} dx$ and $\langle , \rangle$ is the pairing of $H^{-1}(U)$ and $H_0^1(U)$

\textbf {Chapter 6.2 Existence of Weak Solutions} 

\textbf {6.2.1 Lax-Milgram Theorem} \\ We now introduce a fairly simple abstract principle from linear functional analysis, which will later in 6.2.2 provide in certain circumstances the existence and uniqueness of a weak solution to our boundary value problem.

\textbf {Theorem 1 (Lax-Milgram Theorem)} Assume that
\begin{center}
$B : H x H \to R$
\end{center}
is a bilinear mapping, for which there exists constants $\alpha, \beta > 0$ such that
\begin{center}
$|B[u,v]| \leq \alpha ||u|| ||v|| \tab (u,v \in H)$
\end{center}
and 
\begin{center}
$\beta ||u||^2 \leq B[u,u] \tab (u \in H)$
\end{center}
Finally, let $f: H \to R$ be a bounded linear functional on H. \\
\tab Then there exists a unique element $u \in H$ such that
\begin{center}
(1) \tab $B[u,v] = \langle f, v \rangle$
\end{center}
for all $v \in H$

\textbf {Theorem 2 (Energy estimates)} There exist constants $\alpha, \beta > 0$ and $\gamma \geq 0$ such taht
\begin{center}
$|B[u,v]| \leq \alpha ||u||_{H_{0}^{1}(U)}||v||_{H_{0}^{1}(U)}$
\end{center}
and 
\begin{center}
$\beta ||u||^{2}_{H_{0}^{1}(U)} \leq B[u,u] + \gamma ||u||^{2}_{L^{2}(U)}$
\end{center}
for all $u, v \in H_{0}^{1}(U)$

\textbf {Theorem 3 (First Existence Theorem for weak solutions)} There is a number $\gamma \geq 0$ such that for each
\begin{center}
(7) \tab $\mu \geq \gamma$
\end{center}
and each function
\begin{center}
$f \in L^2(U)$
\end{center}
there exists a unique weak solution $u \in H_0^1(U)$ of the boundary-value problem
\begin{center}
(8) \tab \{ $Lu + \mu u = f$ \tab in U \\
\tab \{ u = 0 \tab on $\partial U$
\end{center}

\textbf {Definitions} (i) The operator L*, the formal adjoint of L, is 
\begin{center}
$L*v := - \sum_{i,j = 1}^n (a^{ij}v_{x_{j}{x_{i}}} - \sum_{i=1}^n b^1 v_{x_{i}} + (c - \sum_{i=1}^n b^i_{x_{i}})v$,
\end{center}
provided $b^i \in C^1 (\bar{U}) (i=1,...,n)$ \\
\tab (ii) The adjoint bilinear form
\begin{center}
$B* : H_0^1(U) x H_0^1(U) \to R$
\end{center}
is defined by 
\begin{center}
$B*[v,u] := B[u,v]$
\end{center}
for all $u, v \in H_0^1 (U)$ \\
\tab (iii) We say that $v \in H_0^1(U)$ is a weak solution of the adjoint problem
\begin{center}
$\{L*v = f$ \tab U \\
\{ v = 0 \tab on $\partial U$
\end{center}
provided
\begin{center}
$B*[v,u] = (f,u)$
\end{center}
for all $u \in H_0^1(U)$

\textbf {Theorem 4 (Second Existence Theorem for weak solutions)} \\ \tab (i) Precisely one of the following statements holds either:
\begin{center}
($\alpha)$ \tab \{for each $f \in L^2(U)$ there exists a unique weak solution u of the boundary-value problem

(10) \tab \{ \{Lu = f \tab in U \\
(10) \tab \{ \{u = 0 \tab on $\partial U$
\end{center}
\tab or else 
\begin{center}
($\beta$) \tab \{ there exists a weak solution $u \neq 0$ of the homogeneous problem \\
\tab (11) \tab \{ Lu = 0 \tab in U \\
\tab (11) \tab \{ u = 0 \tab on $\partial U$
\end{center}
\tab (ii) futhermore, should assertion ($\beta$) hold, the dimention of the subspace $N \subset H_0^1(U)$ of weak solutions of (11) is finite and equals the dimensions of the subspace $N* \subset H_0^1(U)$ of weak solutions of 
\begin{center}
(12) \tab \{ L*v = 0 \tab in U \\
\tab \{ v = 0 \tab on $\partial U$
\end{center}
Finally, the boundary-value problem (10) has a weak solution if and only if
\begin{center}
$(f,v) = 0$ \tab for all $v \in N*$
\end{center}
The dichotomy $(\alpha), (\beta)$ is the Fredholm alternative.

\textbf {Theorem 5 (Third Existence THeorem for weak solutions) } \\
\tab (i) There exists an at most countable set $\sum \subset R$ such that the boundary-value problem 
\begin{center}
(24) \tab \{ Lu = $\lambda u + f$ \tab in U \\
(24) \tab \{ u = 0 \tab on $\partial U$
\end{center}
has a unique weak solution for each $f \in L^2(U)$ if and only if $\lambda \notin \sum$ \\ (ii) If $\sum$ is infinite, then $\sum = \{\lambda_k\}_{k=1}^{\infty}$, the values of a nondecreasing sequence with 
\begin{center}
$\lambda_k \to + \infty$
\end{center}

\textbf {Definition} We call $\sum$ the (real) spectrum of the operator L.

\textbf {Theorem 6 (Boundedness of the inverse)} If $\lambda \notin \sum$, there exists a constant C such that 
\begin{center}
(29) \tab $||u||_{L^{2}(U)} \leq C||f||_{L^{2}(U)}$,
\end{center}
whenever $f \in L^2(U)$ and $u \in H_0^{1}(U)$ is the unique weak solution of
\begin{center}
\{ Lu = $\lambda u + f$ \tab in U \\
\{ u = 0 \tab on $\partial U$
\end{center}
The constant C depends only on $\lambda, U$ and the coefficients of L. \\
This constant will blow up if $\lambda$ approaches an eigenvalue .


\textbf {Chapter 6.3 Regularity}

\textbf {Theorem 1 (Interior $H^2$-regularity)} Assume \begin{center}
$a^{ij} \in C^1 (U), b^i, c \in L^{\infty} (U) \tab  (i, j = 1,..., n)$
\end{center}
and 
\begin{center}
$f \in L^2 (U)$
\end{center}
Suppose furthermore that $u \in H^1 (U)$ is a weak solution of the elliptic PDE
\begin{center}
Lu = f \tab in U
\end{center}
Then
\begin{center}
$u \in H_{loc}^2 (U)$;
\end{center}
and for each open subset $V \subset \subset U$ we have the estimate 
\begin{center}
$||u||_{H^{2}(V)} \leq C(||f||_{L^{2}(U)} + ||u||_{L^{2}(U)})$.
\end{center}
the constant C depending only on V, U and the coefficients of L.

\textbf {Theorem 2 (Higher Interior regularity)} Let m be a nonnegative integer, and assume 
\begin{center}
$a^{ij}. b^{i}, c \in C^{m+1}(U) \tab (i, j = 1,...,n)$
\end{center}
and 
\begin{center}
$ f \in H^{m}(U)$
\end{center}
Suppose $u \in H^1(U)$ is a weka solution of the elliptic PDE
\begin{center}
Lu = f \tab in U
\end{center}
Then 
\begin{center}
$u \in H_{loc}^{m+2}(U)$
\end{center}
and for each $V \subset \subset U$ we have the estimate
\begin{center}
(28) \tab $||u||_{H^{m+2}(V)} \leq C(||f||_{H^{m}(U)} + ||u||_{L^{2}(U)})$
\end{center}
the constanc C depending only on m, U, V and the coefficients of L.

\textbf {Theorem 3 (Infinite differentiability in the interior)} Assume 
\begin{center}
$a^{ij}, b^i, c \in C^{\infty} (U) \tab (i,j = 1,...,n)$
\end{center}
and 
\begin{center}
$f \in C^{\infty}(U)$
\end{center}
Suppose $u \in H^1(U)$ is a weak solution of the elliptic PDE
\begin{center}
Lu = f \tab in U
\end{center}
Then 
\begin{center}
$u \in C^{\infty}(U)$
\end{center}

\textbf {Theorem 4 (Boundary $H^2$-regularity)} Assume 
\begin{center}
$a^{ij} \in C^1(\bar{U}), b^i, c \in L^{\infty}(U) \tab (i,j =1,...,n)$
\end{center}
and 
\begin{center}
$f \in L^2(U)$
\end{center}
Suppose that $u \in H_0^1(U)$ is a weak solution of the elliptic boundary-value problem
\begin{center}
(40) \tab \{ Lu = f \tab in U \\
(40) \tab \{ u = 0 \tab $\partial U$
\end{center}
Assume finally
\begin{center}
(41) \tab $\partial U$ is $C^2$
\end{center}
Then
\begin{center}
$u \in H^2(U)$ 
\end{center}
and we have the estimate
\begin{center}
$(42) \tab ||u||_{H^{2}(U)} \leq C(||f||_{L^{2}(U)} + ||u||_{L^{2}(U)})$
\end{center}
the constant C depending only on U and the coefficients of L.

\textbf {Theorem 5 (Higher boundary regularity)} Let m be a nonnegative integer, and assume 
\begin{center}
(72) \tab $a^{ij}, b^i, c \in C^{m+1}(\hat{U}) \tab (i, j = 1,...,n)$
\end{center}
and 
\begin{center}
(73) \tab $f \in H^m(U)$
\end{center}
Suppose that $u \in H_0^1(U)$ is a weak solution of the boundary-value problem
\begin{center}
(74) \tab \{ Lu = f \tab in U \\
\tab \{ u = 0 \tab on $\partial U$
\end{center}
Assume finally
\begin{center}
(75) \tab $\partial U$ is $C^{m+2}$
\end{center}
Then 
\begin{center}
(76) \tab $u \in H^{m+2}(U)$
\end{center}
and we have the estimate
\begin{center}
(77) \tab $||u||_{H^{m+2}(U)} \leq C(||f||_{H^{m}(U)} + ||u||_{L^{2}(U)}$
\end{center}
the constant C depending only on m, U and the coefficients of L.

\textbf {Theorem 6 (Infinite differentiability up to the boundary)} Assume 
\begin{center}
$a^{ij}, b^i, c \in C^{\infty}(\bar{U}) \tab (i, j = 1,...,n)$
\end{center}
and 
\begin{center}
$f \in C^{\infty}(\bar{U})$
\end{center}
Suppose $u \in H_0^1(U)$ is a weak solution of the boundary-value problem
\begin{center}
$\{Lu = f \tab$ in U \\
\{ u = 0 \tab on $\partial U$
\end{center}
Assume also that $\partial U$ is $C^{\infty}$. Then
\begin{center}
$ u \in C^{\infty}(\bar{U})$
\end{center}

\textbf {Chapter 6.4 Maximum principles}

\textbf {Theorem 1 (Weak maximum principle)} Assume $u \in C^{2}(U) \cap C(\bar{U})$ and 
\begin{center}
c = 0 \tab in U
\end{center}
\tab (i) If 
\begin{center}
$Lu \leq 0$ \tab in U
\end{center}
then 
\begin{center}
$max_{\bar{U}}u = max_{\partial U} u$
\end{center}
\tab (ii) If 
\begin{center}
$Lu \geq 0$ \tab in U
\end{center}
then
\begin{center}
$min_{\bar{U}} u = min_{\partial U} u$
\end{center}

\textbf {Theorem 2 (Weak maximum principle for $c \geq 0$)} Assume $u \in C^2(U) \cap C(\bar{U})$ and 
\begin{center}
$c \geq 0$ \tab in U
\end{center}
\tab (i) If
\begin{center}
$Lu \leq 0$ \tab in U
\end{center}
\tab then
\begin{center}
(11) \tab $max_{\bar{U}} u \leq max_{\partial U} u^+$.
\end{center}
\tab (ii) Likewise, if
\begin{center}
$Lu \geq 0 \in U$
\end{center}
then
\begin{center}
(12) \tab $min_{\bar{U}} u \geq - max_{\partial U} u^-$.
\end{center}

\textbf {Lemma (Hopf's Lemma)} Assume $u \in C^2(U) \cap C^1 (\bar{U})$ and 
\begin{center}
c = 0 \tab in U
\end{center}
Suppose further
\begin{center}
$Lu \leq 0$ \tab in U
\end{center}
and there exists a point $x^0 \in \partial U$ such that
\begin{center}
(14) \tab $u (x^0) > u(x)$ \tab for all $x \in U$
\end{center}
Assume finally that U satisfies the interior ball condition at $x^0$; that is, there exists an open ball $B \subset U$ with $x^0 \in \partial B$ \\
\tab (i) Then
\begin{center}
$\frac{\partial u}{\partial v} (x^0) > 0$
\end{center}
where v is the outer unit normal to B at $x^0$ \\
\tab (ii) If
\begin{center}
$c \geq 0$ \tab in U
\end{center}
the same conclusion holds provided 
\begin{center}
$u(x^0) \geq 0$
\end{center}

\textbf {Theorem 3 (Strong maximum principle)} Assume $u \in C^2(U) \cap C(\bar{U})$ and 
\begin{center}
c = 0 \tab in U
\end{center}
Suppose also U is connected, open and bounded \\
\tab (i) If
\begin{center}
$Lu \leq 0$ \tab in U
\end{center}
and u attains its maximum over $\bar{U}$ at an interior point, then
\begin{center}
u is constant within U
\end{center}
\tab (ii) Similarly, if
\begin{center}
$Lu \geq 0$ \tab in U
\end{center}
and u attains its minimum over $\bar{U}$ at an interior point, then
\begin{center}
u is constant within U.
\end{center}

\textbf {Theorem 4 (Strong maximum principle with $c \geq 0$)} Assume $u \in C^2(U) \cap C(\bar{U})$ and
\begin{center}
$c \geq 0$ \tab in U
\end{center}
Suppose also U is connected \\
\tab (i) If
\begin{center}
$Lu \leq 0$ \tab in U
\end{center}
and u attains a nonnegative maximum over $\bar{U}$ at an interior point, then
\begin{center}
u is constant within U
\end{center}
\tab (ii) SImilarly, if
\begin{center}
$Lu \geq 0$ \tab in U
\end{center}
and u attains a nonpositive minimum over $\bar{U}$ at an interior point, then
\begin{center}
u is constant within U
\end{center}

\textbf {Theorem 5 (Harnack's inequality)} Assume $u \geq 0$ is a $C^2$ solution of 
\begin{center}
Lu = 0 \tab in U
\end{center}
and suppose $V \subset \subset U$ is connected. Then there exists a constant C suhc that
\begin{center}
(18) \tab $sup_{V} u \leq C inf_{V} u$
\end{center}
The constant C depends only on V and the coefficients of L

\textbf {Chapter 6.5 Eignevalues and Eigenfunctions}

\textbf {Theorem 1 (Eigenvalues of symmetric elliptic operators)} \\ \tab (i) Each eigenvalue of L is real \\
\tab (ii) Furthermore, if we repeat each eigenvalue according to its (fintie) multiplicity, we have
\begin{center}
$\sum = \{\lambda_{k}\}_{k=1}^{\infty}$
\end{center}
where
\begin{center}
$0 < \lambda_{1} \leq \lambda_{2} \leq \lambda_{3} \leq \dots$
\end{center}
and 
\begin{center}
$\lambda_{k} \to \infty$ \tab as $k \to \infty$
\end{center}
\tab (iii) Finally, there exists an orthonormal basis $\{w_{k}\}_{k=1}^{\infty}$ of $L^2(U)$, where $w_{k} \in H_0^1(U)$ is an eigenfunction corresponding to $\lambda_{k}$
\begin{center}
(4) \tab \{ $Lw_{k} = \lambda_{k} w_{k}$ \tab in U \\
(4) \tab \{$w_k = 0$ \tab on $\partial U$
\end{center}
for k = 1,2,...

\textbf {Definition} We call $\lambda_1 > 0$ the principal eigenvalue of L.

\textbf {Theorem 2 (Variational principle for the principal eigenvalue)} \\ \tab (i) We have
\begin{center}
(5) \tab $\lambda_1 = min\{B[u,u] | u \in H_0^1 (U), ||u||_{L^{2}} = 1\}$
\end{center}
\tab (ii) Furthermore, the above minimum is attained for a function $w_1$, positive within U, which solves
\begin{center}
$\{ Lw_1 = \lambda_1 w_1$ \tab in U \\
\{ $w_1 = 0$ \tab on $\partial U$
\end{center}
\tab (iii) Finally, if $u \in H_0^1(U)$ is any weak solution of 
\begin{center}
$\{ Lu = \lambda_1 u$ \tab in U \\
\{ u = 0 \tab on $\partial U$
\end{center}
then u is a multiple of $w_1$

\textbf {Theorem 3 (Principle eigenvalue for nonsymmetric elliptic operators)} \\ \tab
(i) There exists a real eigenvalue $\lambda_1$ for the operators L, taken with zero boundary conditions, such that if $\lambda \in \mathcal{C}$ is any other eigenvalue, we have
\begin{center}
$Re(\lambda) \geq \lambda_1$
\end{center}
\tab (ii) There exists a corresponding eigenfunction $w_1$, which is positive within U \\
\tab (iii) The eignevalue $\lambda_1$ is simple; that is, if u is any solution of (1), then u is a multiple of $w_1$


















\end{document}